%%%%%%%%%%%%%%%%%%%%%%%%%%%%%%%%%%%%%%%%%%%%%%%%%%%%%%%%%%%%%%%%%%%%%%%%%%%%%%%%
% conclusion.tex:
%%%%%%%%%%%%%%%%%%%%%%%%%%%%%%%%%%%%%%%%%%%%%%%%%%%%%%%%%%%%%%%%%%%%%%%%%%%%%%%%
\chapter{Conclusion}
\label{conclusion_chapter}

Using established graph labeling techniques, namely $\sigma^{+-}$-labelings and $\rho^{+}$-labelings, along with our own original constructions and techniques we have proven that every seven edge forest decomposes $K_{n}$ if and only if $n\equiv 0,1,7,\text{ or }8\pmod{14}$ and $n\geq 14$. We also proved some results on wraparound edge mappings that preserve lengths across higher order complete graphs in the same families and on galaxy decompositions of complete bipartite graphs. Both of these came out of our work on seven edge forest decompositions of complete graphs. There was also a short chapter on programming which included a new graph visualization software and some labeling solvers.

Of course, a natural continuation of this work would be investigating eight edge forests designs. Additionally, the results on wraparound edge mappings and galaxy graph decompositions are very preliminary and there is a lot open in those areas as well. Specifically, developing a labeling that allows for wraparound edges, and investigating galaxy decompositions of complete graphs. With respect to programming, \verb|tikzgrapher| will continue to be improved but perhaps more exciting and useful are the labeling solvers. Very basic constraint programming algorithms were used for the solvers shared in this paper, so creating more efficient labeling solvers is another area that could be explored and could have a big impact in future research on graph decompositions by elminating the need to find labelings by hand.
%%%%%%%%%%%%%%%%%%%%%%%%%%%%%%%%%%%%%%%%%%%%%%%%%%%%%%%%%%%%%%%%%%%%%%%%%%%%%
%%%%%%%%%%%%%%%%%%%%%%%%%%%%%%%%%%%%%%%%%%%%%%%%%%%%%%%%%%%%%%%%%%%%%%%%%%%%%
