\chapter{Additional Results} \label{chap:Additional Results}

We present two main additional results produced by work done on this project. (1) Wraparound edge mappings that preserve lengths and (2) Galaxy graph decompositions of multipartite graphs.

\section{Wraparound edge mappings that preserve lengths}

\textbf{These mappings only apply to $K_{2mt+m}$ and $K_{2mt+(m+1)}$ where $m>1$ is odd. It is also strongly advised to use them sparingly, they are pretty invasive ways to deal with wraparound edges.} This is merely a barebones sort of framework to begin dealing with wraparound edges in labelings. From personal experience, the 'new' short edge mappings are much nicer to deal with, because they will be fixed after the first step. If one has a choice, it is reccomended to use them instead of the 'new' wraparound edges which will change at every step.

A wraparound edge $uv$ in $K_{2mt+m}$, WLOG $u<v$. Now, if $uv$ is a wraparound edge of length $l$ in $K_{3m}$, it's length eventually won't be preserved at some step $t\mapsto t+c$ for $c\geq 1$ in the infinite family $K_{2mt+m}$. So the idea is, maybe we can map the wraparound edge to a (1) a short edge of length $l$ at each step $t\mapsto t+1$ or (2) a wraparound edge of length $l$ at each step $t\mapsto t+1$. The next theorem is something we have alluded to previously, but we prove it anyways.\newpage
\begin{thm} \label{thm:wraplarger}
    In $K_{n}$, $uv$ is a wraparound edge if and only if the absolute difference of its endpoints is greater than the maximal length in $K_{n};\floor{\frac{n}{2}}<|u-v|$. 
    \begin{proof}
        Let $uv$ be an edge in $K_{n}$ via $\ell$ defined previously. If $uv$ is a wraparound edge, then $n-|u-v|<|u-v|$. So then $\frac{n}{2}-\frac{|u-v|}{2}<\frac{|u-v|}{2}$, and therefore $\floor{\frac{n}{2}}\leq \frac{n}{2}<|u-v|$. If $\floor{\frac{n}{2}}\leq \frac{n}{2}<|u-v|$, note that without loss of generality $u<v$. Then necessarily $n-|u-v|=n-(v-u)<v-u=|u-v|$, so then $n<2(v-u)$ and $\floor{\frac{n}{2}}\leq\frac{n}{2}<(v-u)<|u-v|.$
    
        Thus,
        $$\floor{\frac{n}{2}}<|u-v|\Longleftrightarrow\text{ uv is a wraparound edge.}$$
    \end{proof}
    \end{thm}
There is likely a simpler proof for this theorem. However, as time was spent on more important tasks this is what we have to offer. Also we use the notation $uv$ and $(u,v)$ interchangably here depending on the context for the sake of hygeine.

\begin{thm}\label{thm:genwrapmap}
Let us define the following edge length functions

\begin{align*}
    \ell_{n}(u,v)&=
      \begin{cases}
        \min\{|u-v|,n-|u-v|\}, & u,v\in \mathbb{Z}_{n},\\
        \infty, & u=\infty\text{ or }v=\infty,
      \end{cases}\\
    \ell_{n}^{+}(u,v)&=
      \begin{cases}
        u+v\mathbf{\,mod\,}n, & u,v\in \mathbb{N},\\
        u\mathbf{\,mod\,}n,   & u=\infty,\\
        v\mathbf{\,mod\,}n,   & v=\infty.
      \end{cases}
    \end{align*}
\noindent Now, let $t>1$,$c>0$, and let $m>1$ be odd. Lastlly, let $h =2m(t-1)$. For any wraparound edge $(a,b)$ in $K_{21}$ such that $a<b$, we have
\begin{align}
&\ell_{3m}(a,b)=\ell_{2mt+m}(a-ch,b-(c-1)h)=\ell_{2mt+m}(a+(c-1)h,b+ch)\\
&\ell_{3m}(a,b)=\ell_{m}^{+}(a-ch,b-(c-1)h)=\ell_{m}^{+}(a+(c-1)h,b+ch).
\end{align}
That is, these mappings preserve the standard length $\ell$ and additive length $\ell_{m}^{+}$ modulo $m$ of $ab\in E(K_{3m})$ in $K_{2mt+m}$ and in $K_{2mt+(m+1)}$ where we take $K_{2mt+(m+1)}$ to be $\ZZ_{2mt+m}\cup \{\infty\}$.
\end{thm}

\begin{proof}
Since $uv$ is a wraparound edge where $u<v$, $\ell_{3m}(u,v)=3m-|u-v|=3m-(v-u)=3m+u-v$. Let us simply denote this via $\ell_{ab}=3m+a-b$. Now, let $k=\floor{\frac{m}{2}}=\frac{m-1}{2}$ so the maximal length in $K_{3m}$ is $\floor{\frac{3m}{2}}=\floor{\frac{2m+m}{2}}=\floor{\frac{2m}{2}+\frac{m}{2}}=m+\floor{\frac{m}{2}}=m+k$ since $k,m\geq 1$. Suppose $a\geq 2m$. Then, $(2m\geq a<b<3m)\implies (1\leq b-a<m<m+k).$ But then $|a-b|<m+k$, the maximal length and so $ab$ is not a wraparound edge, a contradiction. So $a<b<2m$.\newline

\noindent [Short]: Let $\alpha= a-h,\beta=b\in \ZZ_{2mt+m}$. Note: $2mt+m-h\equiv 2mt+m-2m(t-1)=2m+m=3m$. Therefore, $3m+a\equiv (2mt+m-h)+a\equiv a-h \pmod{2mt+m}$. So then since $1<t$, we have that $3m+a<3m+2m=2m(2)+m\leq 2mt+m$, and so in fact $\alpha = 3m+a$. Recall that $\beta=b<3m<3m+a=\alpha$. So then we have that $|\alpha-\beta|=\alpha-\beta=(3m+a)-b$. Well, $3m<2mt+m$ for $t>1$. So then $\ell_{2mt+m}(\alpha\beta)=\min\{|\alpha-\beta|,2mt+m-|\alpha-\beta|\}=\min\{3m+a-b, 2mt+m+a-b\}=3m+a-b=\ell_{ab}$.\newline

\noindent [Wraparound]: Instead, let $\alpha =a,\beta=b+h\in \ZZ_{2mt+m}$. Clearly, $\alpha=a<b+h=\beta$. So $|\alpha-\beta|=\beta-\alpha=b+h-a$. Recall that $ab$ is a wraparound edge in $K_{3m}$ with $a<b$. So then $|a-b|=b-a>m+k$, the maximal length in $K_{3m}$. So then $|\alpha-\beta|=b-a+h>m+k+h=m+k+2m(t-1)$. Now, the maximal length in $K_{2mt+m}$ is $\floor{\frac{2mt+m}{2}}=mt+k=m+k+m(t-1)$. Well, (i) $(2m(t^{*}-1))_{t^{*}\geq 2}$ and (ii) $(m(t^{*}-1))_{t^{*}\geq 2}$ are both arithmetic sequences with increments $2m$ and $m$, respectively. Both expressions are only equal at $t^{*}=1$, and then afterwards $(i)$ increases faster than $(ii)$. So then we see that $2m(t-1)>m(t-1)$ for all $t>1$, and thus, $|\alpha-\beta|=b-a+h>m+k+h=m+k+2m(t-1)>m+k+m(t-1)$, the maximal length in $K_{2mt+m}$ for $t>1$. So then $\alpha\beta$ is a wraparound edge in $K_{2mt+m}$ of length $2mt-m-|\alpha-\beta|=2mt+m-\beta-\alpha=2mt+m - (b-a+h)= 2mt+m-2m(t-1)+a-b=3m+a-b=\ell_{ab}$.

Now, if we develop the mapped endpoints in [Short] and [Wraparound] by $c-1$, edge lengths $\ell_{2mt+m},\ell_{m}^{+}$ are preserved. Thus, $(6.1)$ and $(6.2)$ are proven.

\end{proof}
Now, although we don't provide any kind of labeling to extend this theorem, we do provide a sort of guardrail that one can implement in their labeling to avoid certain problems. As long as one only applies this mapping to pendant edges, the only thing to worry about is the possibility that an endpoint mapped in a higher order complete graph actually collides with another label(since most labels will presumably not be mapped). This can happen. However the next theorem and corollary are sort of gaurdrails or guides to avoid this issue. The conditions are very terse in their current form, and it would be the goal of future work to simplify these ideas.\newline
\begin{thm}
    Let $t>1,m>1, h = 2m(t-1)$ and $a,b$ be distinct vertices in $K_{3m+m}$ with $a<b$. Next let $\alpha = a-h,\beta = b+h\in V(K_{2mt+m})$, then:
    $$b-a\neq m\text{ or }b\not\equiv a\Mod{m}\Rightarrow \alpha\neq\beta.$$

    \begin{proof}
        Recall that since $a,b\in \ZZ_{2m+m}$ and are distinct, $1\leq b-a<3m$. If $b-a\neq m,$ suppose $\alpha = \beta = a-h=b+h$. Then $b-a\equiv -2h\equiv -4m(t-1)\Mod{2mt+m}$. Well,  $-4m \equiv 2mt+m-4m\equiv 2mt-3m\equiv m(2t-3)\Mod{2mt+m}$. So if $t=2$, $b-a\equiv m(2((2)-3)((2)-1))\equiv m\Mod{2m(2)+m}$ and so $b-a = 7$ or $b-a= m+2m(2)+m = 5m>3m$, both contradictions. So $\alpha\neq \beta$.
        
        If $t>2$, since $t=3\Rightarrow -4m\equiv m(2(2)-3) = 7m-4m\equiv 3m\Mod{2m(3)+m}$ and the sequence $(\floor{m(2n-3)})_{n>2}$ is strictly increasing, $b-a\geq 5m>3m$ for all $t>2$, a contradiction. So $\alpha\neq \beta$.

        Finally, if $b\not\equiv a\Mod{m}$, suppose $\alpha = \beta$. Then $a-h\equiv b-h \Mod{2mt+m}$ and $b-a\equiv -2h\equiv -4m(t-1)\Mod{2mt+m}$. so then $b-a\equiv 0\Mod{m}$ since $m|-4m(t-1)$ and $m|2mt+m.$ But then $b\equiv a\Mod{m}$, a contradiction. So $\alpha\neq \beta$, and the statement is proven.
        
    \end{proof}
\end{thm}

\begin{corollary}\label{thm:gennewvertices}
    Let $t>1,m>1,h=2m(t-1)$ and $ab$ be a wraparound edge in $K_{2m+m}$ such that $a<b$. Then in $K_{2mt+m}:\;a-h\geq 3m$ and $b+h>3m$. Next, let $u,v\in V(K_{2m+m})$ with $v\in [u]_{m}$. If $t=2$, then $|u-v|\neq 2m\Longleftrightarrow u\pm h\neq v$ and $v\pm h \neq u$. If $t>2$, $u\pm h\neq v$ and $v\pm h\neq u$ in $K_{2mt+m}$.
\end{corollary}
\begin{proof}
    Let $k=\frac{m}{2}$ if $m$ is even and $k=\frac{m-1}{2}$ if $m$ is odd. In the proof of Theorem \ref{thm:genwrapmap} it is shown that $a-h=3m+a$, so then since $0\leq a<3m$, $3m\leq 3m+a=a-h$ in $K_{2mt+m}$. Now, suppose $b\leq m$. Then since $a<b\leq m$ and $|a-b|=b-a$, necessarily $1\leq |a-b|\leq m<m+k$, the maximal length in $K_{3m}$. But then by Theorem \ref{thm:wraplarger}, $ab$ is not a wraparound edge, a contradiction. So $b>m$. Therefore $b+h>m+h\geq 3m$.

    Next if $t=2$, then $h=2m$. So if $u+h = v$ or $u-h=v$, $|u-v|=h$. On the other hand if $|u-v|=h$, then $u-v=h$ or $u-v=-h$ so $u+h = v$ or $u-h=v$. So the contrapositive holds and the statement is proven for $t=2.$

    Lastly if $t>2$, then recall that $2mt+m=3m+2m(t-1)=3m+h$ so then since $0\leq u,v<3m$, $u+h,v+h<3m+h=2mt+m$. So then $u+h,v+h\not\in \ZZ_{3m}$ and so necessarily $u+h\neq v$ and $v+ h\neq u$. Now, since $2mt+m-h=3m$, $2mt+m+(v-h)\equiv 3m+v\Mod{2mt+m}$ and similarly $2mt+m+(u-h)\equiv 3m+u\Mod{2mt+m}$, we have that $u-h,v-h$ are simply $3m+u$ and $3m+v$ in $K_{2mt+m}$. Well, since $0\leq u,v<3m$ and $2m(3)+m=7m\leq 2mt+m$ we must have that $3m\leq 3m+u,3m+v<3m+3m=6m<7m\leq 2mt+m$. So then $u-h,v-h\not\in V(K_{3m})$ and therefore necessarily $u-h\neq v$ and $v-h\neq u$ and the statement is proven.

\end{proof}
    This concludes the results of this section. The next section is a fun result that came as a result of dealing with the exceptional graph in Chapter \ref{chap:special case}.

\section{Galaxy Decompositions of Complete Multipartite Graphs}

Chapter \ref{chap:special case}, we deal with





